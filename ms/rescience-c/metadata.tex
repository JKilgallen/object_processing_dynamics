% DO NOT EDIT - automatically generated from metadata.yaml

\def \codeURL{https://github.com/JKilgallen/object_processing_dynamics}
\def \codeDOI{https://doi.org/10.5281/zenodo.17478699}
\def \codeSWH{}
\def \dataURL{https://purl.stanford.edu/bq914sc3730}
\def \dataDOI{}
\def \editorNAME{}
\def \editorORCID{}
\def \reviewerINAME{}
\def \reviewerIORCID{}
\def \reviewerIINAME{}
\def \reviewerIIORCID{}
\def \dateRECEIVED{01 November 2018}
\def \dateACCEPTED{}
\def \datePUBLISHED{}
\def \articleTITLE{A [¬Re]presentational Similarity Analysis of the Dynamics of Object Processing Using Single-Trial EEG Classification}
\def \articleTYPE{Replication}
\def \articleDOMAIN{Computational Neuroscience}
\def \articleBIBLIOGRAPHY{bibliography.bib}
\def \articleYEAR{2025}
\def \reviewURL{}
\def \articleABSTRACT{The recognition of object categories in the human brain has been investigated with electroencephalography (EEG) using classification-based Representational Similarity Analysis (RSA). In the target study, single-trial confusion matrices from a multi-class classifier were used to derive representational dissimilarity matrices (RDMs) and to examine the representational structure of object categories as well as the spatio-temporal dynamics of those representations. While prior work has identified a confound affecting the decoding analyses of the target study, the extent to which this issue affects its central claims remains unclear. Furthermore, our systematic review of the target study identified four additional issues which limit the validity of its findings. First, the principal components used to reduce the dimensionality of the data prior to classification are derived using the entire feature matrix for each subject, thereby ensuring that features of test data are leaked to training data. Second, the metric used to determine the dissimilarity of representations is invalid as it fails to satisfy the non-negativity principle. The assessment of statistical significance is suspect as the study uses the binomial test to compare the number of correct predictions to chance level. However, each test set contains multiple responses to each stimulus, and consequently the assumption of independence between outcomes does not hold. Fourth, the validity of the inference used to assess the significance of their findings is further compromised by the large number of hypothesis tests conducted without appropriate correction for multiple comparisons. In our replication attempt, we implement an analysis pipeline which mitigates these limitations while remaining consistent with the target study in all other respects. We demonstrate that, while the majority of the findings reported in the target study can be reproduced under these limitations, they are not substantiated by a more robust analyses pipeline.
Therefore, we attempted to replicate the findings of the target study by implementing an analysis pipeline which mitigates this issue, while remaining faithful to the original in all other respects. However, the confound affecting the analyses cannot be mitigated for three of the five decoding tasks performed in the target study. Our replication attempt highlights the importance of a standard robust methodology to both multi-class electroencephalography decoding, and classification based RSA.}
\def \replicationCITE{Kaneshiro B, Perreau Guimaraes M, Kim H-S, Norcia AM, Suppes P (2015) A Representational Similarity Analysis of the Dynamics of Object Processing Using Single-Trial EEG Classification. PLoS ONE 10(8) e0135697.}
\def \replicationBIB{kaneshiro_representational_2015}
\def \replicationURL{https://journals.plos.org/plosone/article?id=10.1371/journal.pone.0135697}
\def \replicationDOI{10.1371/journal.pone.0135697}
\def \contactNAME{Jeffrey Mark Siskind}
\def \contactEMAIL{qobi@qobi.org}
\def \articleKEYWORDS{computational-neuroscience, decoding, confound, rsa, python}
\def \journalNAME{ReScience C}
\def \journalVOLUME{4}
\def \journalISSUE{1}
\def \articleNUMBER{}
\def \articleDOI{}
\def \authorsFULL{Jack A. Kilgallen, Barak A. Pearlmutter and Jeffrey Mark Siskind}
\def \authorsABBRV{J.A. Kilgallen, B.A. Pearlmutter and J.M. Siskind}
\def \authorsSHORT{Kilgallen, Pearlmutter and Siskind}
\title{\articleTITLE}
\date{}
\author[1]{Jack A. Kilgallen}
\author[1,2]{Barak A. Pearlmutter}
\author[3]{Jeffrey Mark Siskind}
\affil[1]{Hamilton Institute, Maynooth University, Maynooth, Co. Kildare, Ireland}
\affil[2]{Department of Computer Science, Maynooth University, Maynooth, Co. Kildare, Ireland}
\affil[3]{Elmore Family School of Electrical and Computer Engineering, Purdue University,, West Lafayette, Indiana, USA}
